\documentclass[a4paper, 10pt]{article}
%encoding
%---------------------------------------------------------
\usepackage[utf8]{inputenc}
\usepackage[T1]{fontenc}
%----------------------------------------------------------------------
\usepackage{geometry}
\geometry{left=3cm,right=3cm,top=2.5cm,bottom=2.5cm}
\usepackage[english]{babel}
\usepackage{amsmath}
\usepackage{amsfonts}
\usepackage{amssymb}
\usepackage{array}
\usepackage{graphicx}
\graphicspath{ {img/} }
\usepackage{hyperref}
\usepackage{listings}
\usepackage{verbatim}
\usepackage[shortlabels]{enumitem}
\usepackage[sorting=none]{biblatex}
\bibliography{refs}

\begin{document}
\renewcommand{\abstractname}{\vspace{-\baselineskip}}

\title{Novadoge: The Meme Token Fueling Community Engagement on the Arbitrum Nova Network}
\author{Arbtoshi Novamoto \\ 
\href{novadoge@wp.pl}{novadoge@wp.pl}\\ \url{https://novadoge.net/}}
\date{}


\maketitle
\begin{center}
\parbox{0.8\linewidth}{\noindent \textbf{Abstract.} Novadoge is a cryptocurrency that serves as the native meme token of the Arbitrum nova network, a layer 2 scaling solution for Ethereum. The Novadoge whitepaper outlines the technical specifications and economic model of the Novadoge token, which is primarily designed as a humorous and ironic homage to the Dogecoin meme. Novadoge has no functional utility or governance capabilities, and is intended purely for entertainment and community building. The whitepaper details the token economics and distribution, including the management and use of the token, and the community-focused initiatives and events related to Novadoge. Novadoge is an integral part of the Arbitrum nova network, contributing to its cultural identity and sense of community. Through its playful and lighthearted approach to cryptocurrency, Novadoge and the Arbitrum nova network aim to engage a wider audience and promote the benefits of blockchain technology to a broader audience.}
\end{center}

\

\section{Introduction}\label{introduction}

Since the launch of Bitcoin in 2009, the world of decentralized finance has undergone significant changes, with Ethereum and other blockchain platforms expanding the potential of this technology. However, despite the promise of these innovations, the limitations of first-generation blockchain technology have become increasingly apparent. In particular, issues such as scalability, transaction fees, and network congestion have posed significant challenges to the wider adoption of decentralized finance.

In response to these challenges, the Arbitrum One network emerged as a leading layer 2 scaling solution for Ethereum, which was later succeeded by the Arbitrum Nova network. By enabling fast, low-cost transactions and improving the overall efficiency of the blockchain, Arbitrum Nova has the potential to revolutionize the world of decentralized finance.

At the heart of the Arbitrum Nova network is Novadoge, a cryptocurrency that represents the native meme token of the network. Novadoge resides on the Arbitrum Nova network and is intended to be a playful and lighthearted tribute to the Dogecoin meme, but also serves as an essential component of the network's community and culture.

In this whitepaper, we will explore the technical and economic specifications of Novadoge and its role in powering the Arbitrum Nova network. We will also examine the key features and benefits of the Arbitrum Nova network, including its approach to scaling, transaction processing, and governance.

Through Novadoge and the Arbitrum Nova network, we believe that decentralized finance can reach a wider audience and achieve greater levels of adoption. We hope that this whitepaper will provide a comprehensive overview of the technology and its potential to transform the world of decentralized finance.

\newpage

\section{Why cryptocurrencies exist}\label{Why cryptocurrencies exist}

Cryptocurrencies have emerged as a new form of digital asset that offers a wide range of potential benefits and use cases. At their core, cryptocurrencies are decentralized digital currencies that operate on a blockchain or distributed ledger technology (DLT). Unlike traditional fiat currencies that are issued and managed by central authorities, cryptocurrencies are designed to be peer-to-peer, with no central control or intermediary required to facilitate transactions.

One of the primary reasons that cryptocurrencies exist is to enable secure, decentralized transactions. By operating on a blockchain, cryptocurrencies can be used to transfer value without the need for a trusted third party or intermediary. This is particularly useful in situations where traditional financial institutions may be unavailable or untrustworthy, such as in developing countries or in cases of political instability.

Another key reason for the existence of cryptocurrencies is to provide a means of value storage and protection against inflation. Many cryptocurrencies, such as Bitcoin, have a limited supply, meaning that they cannot be easily inflated or devalued through printing or other means. This makes them a potentially valuable store of value for investors and savers looking to protect their wealth from inflation or other forms of financial instability.

Cryptocurrencies also offer a high level of privacy and anonymity, which can be valuable in certain situations. By operating on a decentralized network, cryptocurrencies can provide a level of privacy and security that is not possible with traditional financial systems. This has made cryptocurrencies popular with individuals and groups that value privacy and anonymity, such as activists, journalists, and whistleblowers.

In addition to these benefits, cryptocurrencies also offer a wide range of potential use cases in various industries. For example, cryptocurrencies can be used to facilitate micropayments and other low-value transactions that would be too expensive or impractical with traditional financial systems. They can also be used to enable cross-border transactions and reduce the costs and complexity of international payments.

Another use case for cryptocurrencies is in the world of decentralized finance (DeFi), which refers to a range of financial applications and services that operate on a blockchain or DLT. DeFi applications enable users to access financial services and products without the need for traditional financial intermediaries, offering greater control and transparency to users. Cryptocurrencies can be used as the means of exchange and value transfer in these applications, providing a secure and decentralized infrastructure for financial transactions.

\begin{figure}[!h]
  \centering
  \includegraphics[width=0.75\linewidth]{4.jpg}
  \end{figure}

Cryptocurrencies can also be used for fundraising and investment, particularly through initial coin offerings (ICOs) and other forms of token sales. These mechanisms enable startups and other organizations to raise capital through the sale of digital tokens, which can represent various types of assets or services. This has created a new avenue for fundraising and investment, particularly for projects and ventures that may not have access to traditional financing channels.

Another reason for the existence of cryptocurrencies is to provide a means of supporting decentralized governance and decision-making. Many blockchain networks, particularly those that use proof-of-stake (PoS) or other consensus mechanisms, require the use of a native cryptocurrency as a means of participating in governance and decision-making. This can enable a more democratic and decentralized approach to network management and operations.

Finally, cryptocurrencies have also emerged as a means of supporting new forms of content creation and distribution. For example, some platforms use cryptocurrencies to incentivize the creation and sharing of digital content, enabling creators to earn rewards and income for their work. This has created new opportunities for creators and artists, particularly in the fields of music, art, and other creative endeavors.

In conclusion, cryptocurrencies exist to enable secure, decentralized transactions, provide a means of value storage and protection against inflation, offer privacy and anonymity, support various use cases in different industries, provide a means of fundraising and investment, enable decentralized governance and decision-making, and support new forms of content creation and distribution. As the world of decentralized finance and blockchain technology continues to evolve, cryptocurrencies are likely to become an even more integral part of the digital ecosystem, offering new solutions and possibilities for individuals, businesses, and governments alike.

Despite the potential benefits of cryptocurrencies, they also face several challenges and limitations. One of the biggest challenges is the issue of volatility and price fluctuations, which can make cryptocurrencies a risky investment or store of value. Another challenge is the issue of scalability and efficiency, particularly with respect to the processing and confirmation of transactions. While various solutions have been proposed to address these issues, there is still significant work to be done to make cryptocurrencies more accessible and user-friendly.

Additionally, cryptocurrencies also face regulatory and legal challenges, particularly with respect to their status and treatment under various jurisdictions. Some governments and financial regulators have expressed concerns about the potential use of cryptocurrencies for illegal activities, such as money laundering and terrorism financing, and have implemented various measures to restrict or monitor their use.

Despite these challenges, the potential benefits of cryptocurrencies are significant and continue to attract interest and investment from a wide range of stakeholders. As new technologies and applications continue to emerge, the role and potential of cryptocurrencies are likely to expand, enabling new forms of innovation and disruption in various industries and sectors.

In conclusion, cryptocurrencies exist to provide a secure, decentralized, and transparent means of transferring value and enabling new forms of financial innovation and interaction. They offer a wide range of benefits and use cases, but also face significant challenges and limitations. As the world of decentralized finance and blockchain technology continues to evolve, the role and potential of cryptocurrencies are likely to expand, shaping the future of finance and commerce. 

\section{Technologies necessary for cryptocurrencies}\label{Technologies necessary for cryptocurrencies}

The emergence of cryptocurrencies was made possible by the convergence of several technologies, including decentralized networks, cryptography, and distributed computing. One of the key technologies that enabled the development of cryptocurrencies is asymmetric cryptography, which enables secure communication and digital signatures.

Asymmetric cryptography, also known as public-key cryptography, uses a pair of keys, one public and one private, to encrypt and decrypt data. The public key can be freely distributed, while the private key must be kept secret. This enables secure communication and digital signatures, ensuring that only the intended recipient can access the data.

Another technology that was instrumental to the emergence of cryptocurrencies is mixnets, which are decentralized networks that enable anonymous communication. Mixnets use a process of routing data through multiple intermediate nodes, or mixers, to conceal the identity of the sender and receiver. This technology inspired the development of more advanced privacy features in cryptocurrencies, such as ring signatures and stealth addresses.

In addition to these technologies, distributed computing and consensus mechanisms, such as proof-of-work and proof-of-stake, are also instrumental to the emergence of cryptocurrencies. These mechanisms enable decentralized networks to reach consensus and prevent double-spending, enabling secure and transparent transactions without the need for a trusted intermediary.

\begin{figure}[!h]
  \centering
  \includegraphics[width=0.75\linewidth]{2.png}
  \end{figure}

The combination of these technologies enabled the creation of the first cryptocurrency, Bitcoin, which was released in 2009. Bitcoin uses a decentralized network and a proof-of-work consensus mechanism to secure transactions and prevent double-spending. It also uses asymmetric cryptography to enable secure communication and digital signatures, and mixnets inspired the development of more advanced privacy features in later cryptocurrencies.

Overall, the emergence of cryptocurrencies was made possible by the convergence of several technologies, including asymmetric cryptography, mixnets, distributed computing, and consensus mechanisms. These technologies enabled the development of decentralized, transparent, and secure digital currencies, offering an alternative to traditional fiat currencies and opening up new possibilities for finance and commerce. As the ecosystem of cryptocurrencies continues to evolve and innovate, it is likely that new technologies and applications will emerge, shaping the future of finance and beyond.

\section{Hash functions}\label{Hash Functions}

In the context of cryptocurrencies, hashing functions play a critical role in ensuring the integrity and security of the blockchain. A hash function is a mathematical function that takes an input of arbitrary length and produces an output of fixed length. Hash functions are designed to be one-way functions, which means it is easy to compute the output from the input, but it is nearly impossible to compute the input from the output.

In the context of cryptocurrencies, hash functions are used to create a unique digital fingerprint, or hash, of a block of transaction data. This hash is then added to the previous block's hash to create a chain of blocks, or a blockchain. Each block in the blockchain contains a hash of the previous block, which ensures that any tampering with a single block will be detected by the network, preventing double-spending and other fraudulent activities.

The most commonly used hash function in cryptocurrencies is the SHA-256 hash function, which was developed by the United States National Security Agency (NSA). The SHA-256 hash function produces a 256-bit hash output and is used in Bitcoin and many other cryptocurrencies.

When a new block of transaction data is added to the blockchain, it is first hashed using the SHA-256 hash function to create a unique digital fingerprint. This hash is then added to the previous block's hash to create a chain of blocks, with each block containing a hash of the previous block. This creates a chain of blocks that cannot be tampered with without detection, as any changes made to a single block will change the hash of that block, which will in turn change the hashes of all subsequent blocks.

Hash functions are also used in the process of mining, which is the process of adding new blocks to the blockchain. In the case of Bitcoin, miners compete to solve a complex mathematical puzzle by hashing the transaction data with their computing power. The first miner to solve the puzzle and find a valid hash can add a new block to the blockchain and earn a reward in the form of newly minted Bitcoin.

Overall, hash functions play a critical role in ensuring the security and integrity of the blockchain in cryptocurrencies. They create a unique digital fingerprint of transaction data, which is used to create a chain of blocks that cannot be tampered with without detection. This enables secure and transparent transactions without the need for a trusted intermediary.

\begin{figure}[!h]
\centering
\includegraphics[width=0.75\linewidth]{14.jpg}
\end{figure}

\section{Public key cryptography}\label{Public key cryptography}

Asymmetric cryptography, also known as public-key cryptography, is a form of cryptography that uses a pair of keys, one public and one private, to encrypt and decrypt data. This technology plays a critical role in ensuring the security and integrity of transactions in cryptocurrencies.

In the context of cryptocurrencies, asymmetric cryptography is used to create secure communication channels and digital signatures. Each user on the network has a pair of keys, one public and one private. The public key can be freely distributed, while the private key must be kept secret. When a user wants to send a transaction, they encrypt the transaction data using the recipient's public key. The recipient can then decrypt the transaction data using their private key.

Asymmetric cryptography also enables the creation of digital signatures, which are used to authenticate the identity of the sender and ensure that the transaction data has not been tampered with. To create a digital signature, the sender encrypts the transaction data using their private key. The recipient can then decrypt the signature using the sender's public key to verify the authenticity of the transaction.

One of the most common algorithms used in asymmetric cryptography is the Elliptic Curve Digital Signature Algorithm (ECDSA), which is used in Bitcoin and many other cryptocurrencies. The ECDSA algorithm uses elliptic curve mathematics to create a unique digital signature that can be verified using the public key.

Overall, asymmetric cryptography plays a critical role in ensuring the security and integrity of transactions in cryptocurrencies. It enables secure communication channels and digital signatures, which are used to authenticate the identity of the sender and ensure that the transaction data has not been tampered with. This helps to prevent double-spending and other fraudulent activities, enabling secure and transparent transactions without the need for a trusted intermediary.

\section{Mixnet}\label{Mixnets}

Mixnets are a type of decentralized network that enable anonymous communication and have inspired the development of more advanced privacy features in cryptocurrencies. In the context of cryptocurrencies, mixnets are used to enable anonymous transactions and protect the privacy of users.

Mixnets work by routing data through a series of intermediate nodes, or mixers, to conceal the identity of the sender and receiver. When a user wants to send a transaction, the transaction data is first encrypted and sent to a mixer. The mixer then combines the transaction data with other encrypted data from other users, making it difficult to trace the transaction back to the sender. The mixer then sends the mixed data to another mixer or directly to the recipient.

The use of mixnets in cryptocurrencies is often referred to as "coin mixing" or "coin tumbling." Coin mixing is a process of mixing the coins of multiple users to obscure the transaction history and prevent the tracing of transactions back to the sender. This helps to protect the privacy and anonymity of users, preventing the identification of individual transactions or user identities.

Some cryptocurrencies have built-in mixnet features, such as Dash, which uses a decentralized network of masternodes to provide coin mixing. Other cryptocurrencies use alternative privacy features, such as ring signatures, which enable a user to sign a transaction using a group of other users' public keys, making it difficult to determine the identity of the actual signer.

Overall, mixnets play a critical role in protecting the privacy and anonymity of users in cryptocurrencies. They enable anonymous communication and transactions, making it difficult to trace transactions back to the sender and prevent the identification of individual transactions or user identities. This helps to prevent fraud, protect user privacy, and enable the use of cryptocurrencies in a wide range of applications and industries.


\section{Ethereum}\label{Ethereum}

\section{Reclaiming Disk Space}\label{reclaiming-disk-space}

\section{Reclaiming Disk Space}\label{reclaiming-disk-space}

Once the latest transaction in a coin is buried under enough blocks, the
spent transactions before it can be discarded to save disk space. To
facilitate this without breaking the block's hash, transactions are
hashed in a Merkle Tree\cite{mer80}\cite{mas99}\cite{hab97}, with only the root included in the block's hash. Old
blocks can then be compacted by stubbing off branches of the tree. The
interior hashes do not need to be stored.

\begin{figure}[!h]
\centering
\includegraphics[width=0.75\linewidth]{reclaiming-disk.png}
\end{figure}

A block header with no transactions would be about 80 bytes. If we
suppose blocks are generated every 10 minutes, 80 bytes * 6 * 24 * 365 =
4.2MB per year. With computer systems typically selling with 2GB of RAM
as of 2008, and Moore's Law predicting current growth of 1.2GB per year,
storage should not be a problem even if the block headers must be kept
in memory.

\newpage

\section{Simplified Payment
Verification}\label{simplified-payment-verification}

It is possible to verify payments without running a full network node. A
user only needs to keep a copy of the block headers of the longest
proof-of-work chain, which he can get by querying network nodes until
he's convinced he has the longest chain, and obtain the Merkle branch
linking the transaction to the block it's timestamped in. He can't check
the transaction for himself, but by linking it to a place in the chain,
he can see that a network node has accepted it, and blocks added after
it further confirm the network has accepted it.

\begin{figure}[!h]
\centering
\includegraphics[width=\linewidth]{spv.png}

\end{figure}

As such, the verification is reliable as long as honest nodes control
the network, but is more vulnerable if the network is overpowered by an
attacker. While network nodes can verify transactions for themselves,
the simplified method can be fooled by an attacker's fabricated
transactions for as long as the attacker can continue to overpower the
network. One strategy to protect against this would be to accept alerts
from network nodes when they detect an invalid block, prompting the
user's software to download the full block and alerted transactions to
confirm the inconsistency. Businesses that receive frequent payments
will probably still want to run their own nodes for more independent
security and quicker verification.

\section{Combining and Splitting
Value}\label{combining-and-splitting-value}

Although it would be possible to handle coins individually, it would be
unwieldy to make a separate transaction for every cent in a transfer. To
allow value to be split and combined, transactions contain multiple
inputs and outputs. Normally there will be either a single input from a
larger previous transaction or multiple inputs combining smaller
amounts, and at most two outputs: one for the payment, and one returning
the change, if any, back to the sender.

\begin{figure}[!h]
\centering
\includegraphics[width=0.3725\linewidth]{combining-splitting.png}

\end{figure}

It should be noted that fan-out, where a transaction depends on several
transactions, and those transactions depend on many more, is not a
problem here. There is never the need to extract a complete standalone
copy of a transaction's history.

\section{Privacy}\label{privacy}

The traditional banking model achieves a level of privacy by limiting access to information to the parties involved and the trusted third party. The necessity to announce all transactions publicly precludes this method, but privacy can still be maintained by breaking the flow of information in another place: by keeping public keys anonymous. The public can see that someone is sending an amount to someone else, but without information linking the transaction to anyone. This is similar to the level of information released by stock exchanges, where the time and size of individual trades, the ``tape'', is made public, but without telling who the parties were.

\begin{figure}[!h]
\centering
\includegraphics[width=0.75\linewidth]{privacy.png}
\end{figure}

As an additional firewall, a new key pair should be used for each
transaction to keep them from being linked to a common owner. Some
linking is still unavoidable with multi-input transactions, which
necessarily reveal that their inputs were owned by the same owner. The
risk is that if the owner of a key is revealed, linking could reveal
other transactions that belonged to the same owner.

\section{Calculations}\label{calculations}

We consider the scenario of an attacker trying to generate an alternate
chain faster than the honest chain. Even if this is accomplished, it
does not throw the system open to arbitrary changes, such as creating
value out of thin air or taking money that never belonged to the
attacker. Nodes are not going to accept an invalid transaction as
payment, and honest nodes will never accept a block containing them. An
attacker can only try to change one of his own transactions to take back
money he recently spent.

The race between the honest chain and an attacker chain can be
characterized as a Binomial Random Walk. The success event is the honest
chain being extended by one block, increasing its lead by +1, and the
failure event is the attacker's chain being extended by one block,
reducing the gap by -1.

The probability of an attacker catching up from a given deficit is
analogous to a Gambler's Ruin problem. Suppose a gambler with unlimited
credit starts at a deficit and plays potentially an infinite number of
trials to try to reach breakeven. We can calculate the probability he
ever reaches breakeven, or that an attacker ever catches up with the
honest chain, as follows\cite{fel57}:

\

\(p\) = {\small probability an honest node finds the next block}

\(q\) = {\small probability the attacker finds the next block}

\(q_z\) = {\small probability the attacker will ever catch up from z blocks behind}

\[
    q_z = 
\begin{cases}
    1               & \text{if } p \leqslant q\\
    \left(q/p\right)^z & \text{if } p > q
\end{cases}
\]

\

Given our assumption that p \textgreater{} q, the probability drops
exponentially as the number of blocks the attacker has to catch up with
increases. With the odds against him, if he doesn't make a lucky lunge
forward early on, his chances become vanishingly small as he falls
further behind.

We now consider how long the recipient of a new transaction needs to
wait before being sufficiently certain the sender can't change the
transaction. We assume the sender is an attacker who wants to make the
recipient believe he paid him for a while, then switch it to pay back to
himself after some time has passed. The receiver will be alerted when
that happens, but the sender hopes it will be too late

The receiver generates a new key pair and gives the public key to the
sender shortly before signing. This prevents the sender from preparing a
chain of blocks ahead of time by working on it continuously until he is
lucky enough to get far enough ahead, then executing the transaction at
that moment. Once the transaction is sent, the dishonest sender starts
working in secret on a parallel chain containing an alternate version of
his transaction.

The recipient waits until the transaction has been added to a block and
z blocks have been linked after it. He doesn't know the exact amount of
progress the attacker has made, but assuming the honest blocks took the
average expected time per block, the attacker's potential progress will
be a Poisson distribution with expected value:

\[\lambda = z \ \frac{q}{p}\]

\

To get the probability the attacker could still catch up now, we
multiply the Poisson density for each amount of progress he could have
made by the probability he could catch up from that point:

\[\sum _{k=0}^\infty \frac{\lambda ^k e^{-\lambda}}{k!} \cdot 
\begin{cases}
    \left(q/p\right)^{(z-p)} & \text{if } k \leqslant z \\
    1                     & \text{if } k > z
\end{cases}
\]

\

Rearranging to avoid summing the infinite tail of the
distribution\ldots{}

\[1 - \sum _{k=0}^z \frac{\lambda ^k e^{-\lambda}}{k!} \left(1 - \left(q/p\right)^{(z-k)}\right)\]

\

Converting to C code\ldots{}

\begin{lstlisting}[language=C]
#include <math.h>
double AttackerSuccessProbability(double q, int z)
{
    double p = 1.0 - q;
    double lambda = z * (q / p);
    double sum = 1.0;
    int i, k;
    for (k = 0; k <= z; k++)
    {
        double poisson = exp(-lambda);
        for (i = 1; i <= k; i++)
            poisson *= lambda / i;
        sum -= poisson * (1 - pow(q / p, z - k));
    }
    return sum;
}
\end{lstlisting}

\

Running some results, we can see the probability drop off exponentially
with z.

\

\begin{verbatim}
q=0.1
z=0 P=1.0000000
z=1 P=0.2045873
z=2 P=0.0509779
z=3 P=0.0131722
z=4 P=0.0034552
z=5 P=0.0009137
z=6 P=0.0002428
z=7 P=0.0000647
z=8 P=0.0000173
z=9 P=0.0000046
z=10 P=0.0000012


q=0.3
z=0 P=1.0000000
z=5 P=0.1773523
z=10 P=0.0416605
z=15 P=0.0101008
z=20 P=0.0024804
z=25 P=0.0006132
z=30 P=0.0001522
z=35 P=0.0000379
z=40 P=0.0000095
z=45 P=0.0000024
z=50 P=0.0000006
\end{verbatim}

Solving for P less than 0.1\%\ldots{}

\begin{verbatim}
P < 0.001
q=0.10 z=5
q=0.15 z=8
q=0.20 z=11
q=0.25 z=15
q=0.30 z=24
q=0.35 z=41
q=0.40 z=89
q=0.45 z=340
\end{verbatim}


\section{Conclusion}\label{conclusion}

We have proposed a system for electronic transactions without relying on
trust. We started with the usual framework of coins made from digital
signatures, which provides strong control of ownership, but is
incomplete without a way to prevent double-spending. To solve this, we
proposed a peer-to-peer network using proof-of-work to record a public
history of transactions that quickly becomes computationally impractical
for an attacker to change if honest nodes control a majority of CPU
power. The network is robust in its unstructured simplicity. Nodes work
all at once with little coordination. They do not need to be identified,
since messages are not routed to any particular place and only need to
be delivered on a best effort basis. Nodes can leave and rejoin the
network at will, accepting the proof-of-work chain as proof of what
happened while they were gone. They vote with their CPU power,
expressing their acceptance of valid blocks by working on extending them
and rejecting invalid blocks by refusing to work on them. Any needed
rules and incentives can be enforced with this consensus mechanism.

\newpage

\printbibliography


\end{document}
